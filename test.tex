\usepackage{amsmath}
\usepackage{graphicx}
\usepackage{url}
\usepackage{duration}
\usepackage{marginnote}
\title{Interrogation}
\targetcg{6}{ème}{Science-Informatique}
\begin{document}
\maketitle

\begin{questions}
\question[1]
Expliquez avec vos mots le noeud du problème.
\begin{solutionordottedlines}[2cm]
lalala
\end{solutionordottedlines}
\marginpar{\duration{2}}

\question[2]
Complétez ces points :
\begin{itemize}
\item A : \teacherorfill{premier}
\item B : \teacherorfill{second}
\end{itemize}
\marginpar{\duration{3}}

\question[1]
Choisissez bien :
\begin{oneparcheckboxes}
\choice oui
\correctchoice non
\end{oneparcheckboxes}
\marginpar{\duration{1}}

\question[1]
Même chose, davantage de possibilités :
\begin{checkboxes}
\choice un
\correctchoice deux
\choice trois
\choice quatre
\end{checkboxes}
\marginpar{\duration{1}}

\teacheronly{Au total, ça fait \totalduration{} minutes.}

\end{questions}
\end{document}
